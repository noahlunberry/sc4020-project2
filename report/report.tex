\documentclass[12pt]{article}
\usepackage{graphicx} % Required for inserting images
\graphicspath{images/} % Direct to "images" folder
\usepackage{subcaption}
\usepackage{caption}
\usepackage{float}
\usepackage{conveniences}
\usepackage{geometry}
\usepackage{parskip}
\usepackage{ragged2e}
\usepackage{amssymb}
\usepackage{titlesec}
\usepackage{fancyhdr}
\usepackage{xcolor}
\usepackage[titles]{tocloft}  % Allows customization of ToC layout
\usepackage{etoolbox}
\usepackage{comment}

\usepackage{hyperref}
\renewcommand{\sectionautorefname}{Section}
\renewcommand{\subsectionautorefname}{Section}
\renewcommand{\subsubsectionautorefname}{Section}

\titleformat{\section}
    {\centering\fontS{18}\bfseries} %Centered, Font 18, Bold
    {\thesection} %No numbering
    {0.5em} %No extra space
    {} %No extra formatting
    [\vspace{20pt}\titlerule\vspace{10pt}]

\begin{document}
\begin{center}
    \includegraphics[width=\linewidth]{NTU_Logo.png}
    \\[1cm]
    \fontS{20}
    \underline{\textbf{SC4020 Data Analytics and Mining}}
    \\[1.5em]
    \fontS{14}
    Academic Year 2025/2026
    \\[1em]
    Semester 1
    \\[2em]
    Group 18
    \\[5em]
    \textbf{
        LUNBERRY NOAH IWATA (N2503869H) \\[1em]
        PIKERINGA ANTONINA DAILA (N2504101A) \\[1em]
        RAHLFS FREDERIC MAURITZ (N2504096K) \\[1em]
        SARAH EMILY ONG XIN WEI (U2440124G) \\[1em]
        }
\end{center}
\pagebreak

\justifying

\pagestyle{fancy}
\fancyhf{}  % Clear default header/footer
\fancyhead[R]{\textcolor{gray}{\nouppercase{\leftmark}}}   % Left header shows current section title
\fancyfoot[C]{\thepage}  % Footer center shows page number

\pagenumbering{roman}

\pagebreak
\renewcommand{\cftdotsep}{0.5}
\renewcommand{\cftsecleader}{\cftdotfill{\cftdotsep}}
\renewcommand{\contentsname}{Table of Contents}  % Set ToC title text
\setlength{\cftbeforesecskip}{10pt}   % Space before sections
\setlength{\cftbeforesubsecskip}{10pt} % Space before subsections
\setlength{\cftbeforesubsubsecskip}{10pt} % Space before subsections
\renewcommand{\cftsecpresnum}{Chapter~} % Adds "Chapter" before section number
\renewcommand{\cftsecaftersnum}{\quad} 
\setlength{\cftsecnumwidth}{6.1em}   %hardcoded in, will die if you have double digit chapters
%\renewcommand{\numberline}[1]{Chapter #1\quad} %old code, works as well but add Chapter to the subsections too which is L

\tableofcontents

\pagebreak
\pagenumbering{arabic}
\section{Task 1: Analysis of Symptom Co-occurrence Patterns}

In this task, we analysed the co-occurrence pattern of different symptoms within disease profiles using the Apriori algorithm.

\subsection{Data Preprocessing}
The dataset is loaded from a CSV file from the Disease Symptom Prediction Dataset. 
It was then cleaned to remove null values. 
Each row represents a transaction. We then extract the symptoms (items) associated with each case from every row (transaction). 

The symptoms are then convereted to lowercase and we remove any whitespace
This ensures that when we store the symptoms, there are no duplicate symptoms within transactions.
Using the \texttt{TransactionEncoder} from the \texttt{mlxtend} library, the transactions are transformed via one-hot encoding. 
This facilitates the implementation of the Apriori algorithm.

\subsection{Implementation of Apriori Algorithm}
The Apriori algorithm is implemented using the \texttt{mlxtend} library, with the goal of mining frequent itemsets from the preprocessed transaction data. 
The minimum support threshold is set to 0.1, meaning that an itemset must appear in at least 10\% of the transactions to be considered frequent. 

Following the discovery of frequent itemsets, the \texttt{association\_rules} function is used to generate association rules with a minimum confidence threshold of 0.6. 
This ensures that only rules with a confidence greater than or equal to 60\% are considered valid. 

The key parameters used are support and confidence, which are essential for filtering out weak rules and itemsets.

\subsection{Results}
The frequent itemsets and association rules discovered using the Apriori algorithm are shown in the table below. 
The frequent itemsets include individual symptoms that appear together in the dataset with a support greater than or equal to 0.1. 
The association rules highlight the relationships between symptoms, where the antecedents (conditions) are linked to the consequents (outcomes), with each rule being evaluated based on its support and confidence.

\begin{table}[ht]
\centering
\begin{tabular}{|l|l|l|l|}
\hline
\textbf{Antecedents}        & \textbf{Consequents}       & \textbf{Support} & \textbf{Confidence} \\ \hline
dark\_urine                 & abdominal\_pain            & 0.110976         & 0.957895           \\ \hline
abdominal\_pain             & loss\_of\_appetite         & 0.132927         & 0.633721           \\ \hline
abdominal\_pain             & vomiting                   & 0.176829         & 0.843023           \\ \hline
yellowing\_of\_eyes         & abdominal\_pain            & 0.114634         & 0.691176           \\ \hline
yellowish\_skin             & abdominal\_pain            & 0.154878         & 0.835526           \\ \hline
\end{tabular}
\caption{Association Rules with Support and Confidence}
\end{table} 

\pagebreak
\section{Task 2: Mining Cancer Feature Patterns}

In this task we looked at the Breast Cancer Wisconsin (Diagnostic) dataset to find simple repeating feature patterns that appear in malignant and benign cases. 
The idea is to turn all the numeric feature values into small categories (low, med, high) and then see what combinations show up the most.

\subsection{Data Preprocessing}
We first load the dataset and drop the \texttt{id} column since it is not useful for the analysis. 
The column \texttt{diagnosis} is encoded where M is 1 (malignant) and B is 0 (benign). 
Most features are numeric so we clean non-numeric columns and remove rows that have missing values.

To make the features easier to compare, we scale them with z-score normalization. 
Then we discretize them into three bins (low, medium, high) using the quantile method so that each group has roughly the same number of samples. 
This makes it easier to see which feature values tend to be high or low for each patient.

\subsection{Building Sequences}
After the data is cleaned, we build a small ordered sequence for each patient. 
For example, if a patient has many “high” values in features like radius or concavity, that patient’s sequence might look like:
\[
\langle \{radius\_mean\_high\}, \{concavity\_mean\_high\}, \{smoothness\_mean\_low\} \rangle
\]
Each sequence keeps at most three itemsets (top features per patient). 
This keeps the data simple and faster to mine.

\subsection{Pattern Mining with GSP}
We use a basic Generalized Sequential Pattern (GSP) algorithm to find patterns that appear often in the malignant and benign groups. 
The minimum support is set to 0.25 so a pattern must appear in at least 25\% of the cases to be kept. 
The maximum pattern length is 2 since longer patterns made the search slow and did not add much insight.

\subsection{Results}
The results show several strong single-feature patterns for both classes. 
The top frequent itemsets are shown below.

\begin{table}[H]
\centering
\begin{tabular}{|l|l|l|l|}
\hline
\textbf{Class} & \textbf{Length} & \textbf{Pattern} & \textbf{Support} \\ \hline
Malignant & 1 & \{radius\_worst\_high\} & 0.844 \\ \hline
Malignant & 1 & \{concave\_points\_worst\_high\} & 0.840 \\ \hline
Malignant & 1 & \{perimeter\_worst\_high\} & 0.840 \\ \hline
Malignant & 1 & \{area\_worst\_high\} & 0.835 \\ \hline
Malignant & 1 & \{concave\_points\_mean\_high\} & 0.816 \\ \hline
Benign & 1 & \{radius\_worst\_high\} & 0.844 \\ \hline
Benign & 1 & \{concave\_points\_worst\_high\} & 0.840 \\ \hline
Benign & 1 & \{perimeter\_worst\_high\} & 0.840 \\ \hline
Benign & 1 & \{area\_worst\_high\} & 0.835 \\ \hline
Benign & 1 & \{concave\_points\_mean\_high\} & 0.816 \\ \hline
\end{tabular}
\caption{Top frequent patterns in malignant and benign cases}
\end{table}

\subsection{Discussion}
We see that features related to size and shape, such as \texttt{radius\_worst}, \texttt{perimeter\_worst}, and \texttt{area\_worst}, often appear with high values in both malignant and benign samples. 
These features describe how large and irregular a cell cluster is. 
The pattern also shows that \texttt{concave\_points} is a major sign of cancer shape irregularity. 

Even though some high features appear in both groups, malignant cases usually have higher intensity and more frequent occurrence of these “high” values. 
So the pattern mining results match what is known in medical literature — tumors that are larger and have more concave shapes tend to be malignant.

\subsection{Conclusion for Task 2}
By turning continuous cancer features into simple levels and using a short GSP search, we can see clear, human-readable patterns in the data. 
This small test shows how sequential pattern mining can help understand which feature combinations are common in cancer diagnosis, without using complex models.

\pagebreak
\section{Task 3: Open Advanced Task}

\pagebreak
\section{Conclusion} % potentially optional

\pagebreak
\section*{Appendix A - Breakdown of Contributions}
\pagenumbering{arabic}
\markboth{Appendix A - Breakdown of Contributions}{}
\renewcommand{\thepage}{A-\arabic{page}}
\addcontentsline{toc}{section}{Appendix A - Breakdown of Contributions}

\textbf{LUNBERRY NOAH IWATA (N2503869H)} \\
Coding, analysis, and write-up for Task 2. \\

\textbf{PIKERINGA ANTONINA DAILA (N2504101A)} \\
Coding, analysis, and write-up for Task 2. \\

\textbf{RAHLFS FREDERIC MAURITZ (N2504096K)} \\
Coding, analysis, and write-up for Task 3. \\

\textbf{SARAH EMILY ONG XIN WEI (U2440124G)} \\
Coding, analysis, and write-up for Task 1. \\

\end{document}