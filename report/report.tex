\documentclass[12pt]{article}
\usepackage{graphicx} % Required for inserting images
\graphicspath{images/} % Direct to "images" folder
\usepackage{subcaption}
\usepackage{caption}
\usepackage{float}
\usepackage{conveniences}
\usepackage{geometry}
\usepackage{parskip}
\usepackage{ragged2e}
\usepackage{amssymb}
\usepackage{titlesec}
\usepackage{fancyhdr}
\usepackage{xcolor}
\usepackage[titles]{tocloft}  % Allows customization of ToC layout
\usepackage{etoolbox}
\usepackage{comment}

\usepackage{hyperref}
\renewcommand{\sectionautorefname}{Section}
\renewcommand{\subsectionautorefname}{Section}
\renewcommand{\subsubsectionautorefname}{Section}

\titleformat{\section}
    {\centering\fontS{18}\bfseries} %Centered, Font 18, Bold
    {\thesection} %No numbering
    {0.5em} %No extra space
    {} %No extra formatting
    [\vspace{20pt}\titlerule\vspace{10pt}]

\begin{document}
\begin{center}
    \includegraphics[width=\linewidth]{NTU_Logo.png}
    \\[1cm]
    \fontS{20}
    \underline{\textbf{SC4020 Data Analytics and Mining}}
    \\[1.5em]
    \fontS{14}
    Academic Year 2025/2026
    \\[1em]
    Semester 1
    \\[2em]
    Group 18
    \\[5em]
    \textbf{
        LUNBERRY NOAH IWATA (N2503869H) \\[1em]
        PIKERINGA ANTONINA DAILA (N2504101A) \\[1em]
        RAHLFS FREDERIC MAURITZ (N2504096K) \\[1em]
        SARAH EMILY ONG XIN WEI (U2440124G) \\[1em]
        }
\end{center}
\pagebreak

\justifying

\pagestyle{fancy}
\fancyhf{}  % Clear default header/footer
\fancyhead[R]{\textcolor{gray}{\nouppercase{\leftmark}}}   % Left header shows current section title
\fancyfoot[C]{\thepage}  % Footer center shows page number

\pagenumbering{roman}

\pagebreak
\renewcommand{\cftdotsep}{0.5}
\renewcommand{\cftsecleader}{\cftdotfill{\cftdotsep}}
\renewcommand{\contentsname}{Table of Contents}  % Set ToC title text
\setlength{\cftbeforesecskip}{10pt}   % Space before sections
\setlength{\cftbeforesubsecskip}{10pt} % Space before subsections
\setlength{\cftbeforesubsubsecskip}{10pt} % Space before subsections
\renewcommand{\cftsecpresnum}{Chapter~} % Adds "Chapter" before section number
\renewcommand{\cftsecaftersnum}{\quad} 
\setlength{\cftsecnumwidth}{6.1em}   %hardcoded in, will die if you have double digit chapters
%\renewcommand{\numberline}[1]{Chapter #1\quad} %old code, works as well but add Chapter to the subsections too which is L

\tableofcontents

\pagebreak
\pagenumbering{arabic}
\section{Task 1: Analysis of Symptom Co-occurrence Patterns}

In this task, we analysed the co-occurrence pattern of different symptoms within disease profiles using the Apriori algorithm.

\subsection{Data Preprocessing}
The dataset is loaded from a CSV file from the Disease Symptom Prediction Dataset. 
It was then cleaned to remove null values. 
Each row represents a transaction. We then extract the symptoms (items) associated with each case from every row (transaction). 

The symptoms are then convereted to lowercase and we remove any whitespace
This ensures that when we store the symptoms, there are no duplicate symptoms within transactions.
Using the \texttt{TransactionEncoder} from the \texttt{mlxtend} library, the transactions are transformed via one-hot encoding. 
This facilitates the implementation of the Apriori algorithm.

\subsection{Implementation of Apriori Algorithm}
The Apriori algorithm is implemented using the \texttt{mlxtend} library, with the goal of mining frequent itemsets from the preprocessed transaction data. 
The minimum support threshold is set to 0.1, meaning that an itemset must appear in at least 10\% of the transactions to be considered frequent. 

Following the discovery of frequent itemsets, the \texttt{association\_rules} function is used to generate association rules with a minimum confidence threshold of 0.6. 
This ensures that only rules with a confidence greater than or equal to 60\% are considered valid. 

The key parameters used are support and confidence, which are essential for filtering out weak rules and itemsets.

\subsection{Results}
The frequent itemsets and association rules discovered using the Apriori algorithm are shown in the table below. 
The frequent itemsets include individual symptoms that appear together in the dataset with a support greater than or equal to 0.1. 
The association rules highlight the relationships between symptoms, where the antecedents (conditions) are linked to the consequents (outcomes), with each rule being evaluated based on its support and confidence.

\begin{table}[ht]
\centering
\begin{tabular}{|l|l|l|l|}
\hline
\textbf{Antecedents}        & \textbf{Consequents}       & \textbf{Support} & \textbf{Confidence} \\ \hline
dark\_urine                 & abdominal\_pain            & 0.110976         & 0.957895           \\ \hline
abdominal\_pain             & loss\_of\_appetite         & 0.132927         & 0.633721           \\ \hline
abdominal\_pain             & vomiting                   & 0.176829         & 0.843023           \\ \hline
yellowing\_of\_eyes         & abdominal\_pain            & 0.114634         & 0.691176           \\ \hline
yellowish\_skin             & abdominal\_pain            & 0.154878         & 0.835526           \\ \hline
\end{tabular}
\caption{Association Rules with Support and Confidence}
\end{table} 

\pagebreak
\section{Task 2: Mining Cancer Feature Patterns}

\pagebreak
\section{Task 3: Open Advanced Task}

\pagebreak
\section{Conclusion} % potentially optional

\pagebreak
\section*{Appendix A - Breakdown of Contributions}
\pagenumbering{arabic}
\markboth{Appendix A - Breakdown of Contributions}{}
\renewcommand{\thepage}{A-\arabic{page}}
\addcontentsline{toc}{section}{Appendix A - Breakdown of Contributions}

\textbf{LUNBERRY NOAH IWATA (N2503869H)} \\
Coding, analysis, and write-up for Task 2. \\

\textbf{PIKERINGA ANTONINA DAILA (N2504101A)} \\
Coding, analysis, and write-up for Task 2. \\

\textbf{RAHLFS FREDERIC MAURITZ (N2504096K)} \\
Coding, analysis, and write-up for Task 3. \\

\textbf{SARAH EMILY ONG XIN WEI (U2440124G)} \\
Coding, analysis, and write-up for Task 1. \\

\end{document}